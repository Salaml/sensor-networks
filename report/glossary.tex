\usepackage[
	nonumberlist,%  keine Seitenzahlen anzeigen
	acronym,%       ein Abkürzungsverzeichnis erstellen
	toc,%           Einträge im Inhaltsverzeichnis
	nopostdot,%     kein Punkt am Ende der Beschreibung
	automake%
]{glossaries}% Glossar

\newglossary[slg]{symbolslist}{syi}{syg}{Symbolverzeichnis}% eigenes Symbolverzeichnis erstellen

\renewcommand{\glsdescwidth}{0.7\textwidth}% Breite bei Typ long erhöhen

% Befehl für Eintrag, der sowohl im Glossar als auch im Abkürzungsverzeichnis auftaucht
\newcommand{\newglossaryacronym}[4]{
	\newglossaryentry{#1}{
		name=\glslink{#2}{#3},
		text=#3,
		description={#4},
		sort={#3}% nach Text sortieren
	}
	\newacronym[description={\glslink{#1}{#3}}]{#2}{#2}{#3}
}

\makeglossaries% Glossar erzeugen

% Symbole
\newglossaryentry{sym.Rc}{
	name=$R_c$,
	description={Chiprate der Signalübertragung},
	sort=rc, type=symbolslist
}

\newglossaryentry{sym.B}{
	name=$B$,
	description={Bandbreite der Signalübertragung},
	sort=b, type=symbolslist
}

\newglossaryentry{sym.fmax}{
	name=$f_{max}$,
	description={maximale Frequenz des \glspl{Chirp} bei der Signalübertragung},
	sort=fmax, type=symbolslist
}

\newglossaryentry{sym.fmin}{
	name=$f_{min}$,
	description={minimale Frequenz des \glspl{Chirp} bei der Signalübertragung},
	sort=fmin, type=symbolslist
}

\newglossaryentry{sym.Rs}{
	name=$R_s$,
	description={Symbolrate der Signalübertragung},
	sort=rs, type=symbolslist
}

\newglossaryentry{sym.SF}{
	name=$SF$,
	description={\acrlong{SF} der Signalübertragung},
	sort=sf, type=symbolslist
}

\newglossaryentry{sym.Rb}{
	name=$R_b$,
	description={Bitrate der Signalübertragung},
	sort=rb, type=symbolslist
}

\newglossaryentry{sym.CR}{
	name=$CR$,
	description={Code Rate für Fehlerkorrektur},
	sort=cr, type=symbolslist
}

\newglossaryentry{sym.Rbnet}{
	name=$R_{b,netto}$,
	description={Nettobitrate der Signalübertragung},
	sort=rbnet, type=symbolslist
}

% Beispiel: Eintrag, der sowohl im Glossar als auch im Abkürzungsverzeichnis auftaucht:
\newglossaryacronym{ID für Verweis auf Glossar}{Abkürzung / ID für Verweis Abk.-Verzeichnis}{langer Name}{Beschreibung im Glossar}

\newglossaryacronym{snr}{SNR}{Signal-Rausch-Verhältnis}{Maß für die Qualität des Nutzsignals bei einer Datenübertragung}

\newglossaryacronym{lora}{LoRa}{Long Range}{proprietärer Funkstandard mit Fokus auf hoher Reichweite bei geringem Energiebedarf}

\newglossaryacronym{sf}{SF}{Spreading Factor}{steuert Übertragunsrate und \gls{Airtime} bei \acrshort{LoRa}-Übertragung}

\newglossaryacronym{srd}{SRD}{Short Range Device Band}{lizenzfreies Frequenzband für Übertragungen über kurze Reichweiten bzw. geringer Leistung}

\newglossaryacronym{ism}{ISM}{{Industrial, Scientific and Medical Band}}{lizenzfrei nutzbare Frequenzbereiche}

\newglossaryacronym{lorawan}{LoRaWAN}{Long Range Wide Area Network}{auf Energiesparsamkeit bei der Datenübertragung ausgelegtes Weitverkehrsnetzwerk unter Nutzung des Funkstandards \acrshort{LoRa}}

\newglossaryacronym{ttn}{TTN}{The Things Network}{weltweites, offenes und communitybasiertes \acrshort{LoRaWAN}}

\newglossaryacronym{abp}{ABP}{Activation by Personalization}{Aktivierung des \gls{Node} im \acrshort{LoRaWAN} mit vorberechneten Sitzungsschlüsseln}

\newglossaryacronym{otaa}{OTAA}{Over-The-Air-Activation}{Aktivierung des \gls{Node} im \acrshort{LoRaWAN} über Join-Prozess mit dynamischer Berechnung Sitzungsschlüssel}

\newglossaryacronym{crc}{CRC}{Cyclic Redundancy Check}{Prüfwert für Fehlererkennung oder -korrektur}

\newglossaryacronym{adr}{ADR}{Adaptive Data Rate}{Verfahren zur Optimierung von \gls{Airtime}, Energiebedarf und Reichweite bei \gls{LoRaWAN}}

\newglossaryentry{Gateway}{
	name=Gateway,
	description={Schnittstelle zwischen \acrshort{LoRa}-Kommunikation und Netzwerkserver im \acrshort{LoRaWAN}}
}

\newglossaryentry{Node}{
	name=Node,
	description={Endgerät im \acrshort{LoRaWAN}, kann per \acrshort{LoRa} Daten senden und empfangen}
}

\newglossaryentry{Uplink}{
	name=Uplink,
	description={Nachricht von \gls{Node} via \gls{Gateway} an \acrshort{LoRaWAN}-Anwendung}
}

\newglossaryentry{Downlink}{
	name=Downlink,
	description={Nachricht von \acrshort{LoRaWAN}-Anwendung via \gls{Gateway} an \gls{Node}}
}

\newglossaryentry{Airtime}{
	name=Airtime,
	description={Dauer, welche zum Senden einer Nachricht benötigt wird}
}

\newglossaryentry{Chirp}{
	name=Chirp,
	description={Signal steigender oder fallender Frequenz mit konstanter Amplitude}
}

\newglossaryentry{Modulation}{
	name=Chirp Spread Spectrum,
	description={Modulationsverfahren unter Verwendung von \glspl{Chirp}}
}

\newglossaryentry{Netserver}{
	name=Netzwerkserver,
	description={zentrales Element für Routing von Nachrichten zwischen \glspl{Gateway} und \gls{Appserver} im \gls{LoRaWAN}}
}

\newglossaryentry{Appserver}{
	name=Applikationsserver,
	description={Schnittstelle zwischen \gls{LoRaWAN} und Anwendungen}
}

\newglossaryentry{Joinserver}{
	name=Join-Server,
	description={Server für Schlüsselverwaltung und -verteilung im \gls{LoRaWAN}}
}

\newglossaryentry{Payload}{
	name=Payload,
	description={Nutzdaten zwischen verschiedenen Objekten, das heißt ohne Steuer- und Protokollinformationen}
}