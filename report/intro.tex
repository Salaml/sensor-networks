\chapter{Einführung}
Das Internet of Things (IoT) ist nicht nur in der IT-Branche ein aktuelles Thema sondern dringt auch immer mehr in die Welt von Otto Normalverbraucher vor.
Das Einstellen der Heizkörpertemperaturen, Schalten von Steckdosen oder Protokollieren von Messwerten wie Temperatur oder Feinstaubgehalt der Luft über das Internet ist für viele Menschen bereits Normalität.
Das \gls{TTN} bildet einen für jeden nahezu frei nutzbaren Zugang zum Internet of Things mit großer räumlicher Abdeckung.

Grundlage des TTN bildet das Ökosystem aus LoRa und LoRaWAN.
LoRa ist dabei eine Funktechnologie, die zur Übertragung von Daten über große Distanzen mit wenig Energieaufwand verwendet wird. 
Mittels LoRaWAN können Geräte, die LoRa beherrschen, in ein Netzwerk eingebunden werden und damit Teil des Internet of Things werden.
Kernelement von LoRa und LoRaWAN ist dabei ein möglichst niedriger Energiebedarf der Endgeräte.

\image{lorawan-architecture-simple}{img.arch-simple}{0.9\textwidth}{Architektur von LoRaWAN (vereinfacht)}{Architektur von LoRaWAN (vereinfacht)}{bearbeitet aus Zerynth srl, https://www.zerynth.com/wp-content/uploads/2017/05/lorawan-architecture.jpg}
Anhand von \autoref{img.arch-simple} soll die Abgrenzung von LoRa und LoRaWAN verdeutlicht werden.
Über LoRa wird lediglich zwischen den Endgeräten und den Gateways kommuniziert, den Schnittstellen zum weiteren Netzwerk.
Jegliche weitere Kommunikation wird über das Internet abgewickelt.
Die gesamte Abbildung stellt das LoRaWAN dar, LoRa ist nur die Funktechnologie die auf der untersten Netzebene genutzt wird.

In den folgenden Kapiteln wird zunächst näher auf LoRa als Grundlage der Kommunikation eingegangen, anschließend wird die Vernetzung der Geräte zum LoRaWAN erläutert.
