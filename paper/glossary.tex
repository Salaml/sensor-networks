\usepackage[
    nonumberlist,%  keine Seitenzahlen anzeigen
    acronym,%       ein Abkürzungsverzeichnis erstellen
    toc,%           Einträge im Inhaltsverzeichnis
    automake%
]{glossaries}% Glossar

\newglossary[slg]{symbolslist}{syi}{syg}{Symbolverzeichnis}% eigenes Symbolverzeichnis erstellen

\renewcommand*{\glspostdescription}{}% Punkt am Ende der Beschreibung deaktivieren

% Befehl für Eintrag, der sowohl im Glossar als auch im Abkürzungsverzeichnis auftaucht
\newcommand{\newglossaryacronym}[4]{
  \newglossaryentry{#1}{
    name=\glslink{#2}{#3},
    text=#3,
    description={#4},
    sort={#3}% nach Text sortieren
  }
  \newacronym[description={\glslink{#1}{#3}}]{#2}{#2}{#3}
}

\makeglossaries% Glossar erzeugen

%Befehle für Symbole
\newglossaryentry{symb:Pi}{
name=$\pi$,
description={Die Kreiszahl.},
sort=symbolpi, type=symbolslist
}
\newglossaryentry{symb:Phi}{
name=$\varphi$,
description={Ein beliebiger Winkel.},
sort=symbolphi, type=symbolslist
}

% Beispiel: Eintrag, der sowohl im Glossar als auch im Abkürzungsverzeichnis auftaucht:
\newglossaryacronym{ID für Verweis auf Glossar}{Abkürzung / ID für Verweis Abk.-Verzeichnis}{langer Name}{Beschreibung im Glossar}

\newglossaryacronym{snr}{SNR}{Signal-Rausch-Verhältnis}{Maß für die Qualität des Nutzsignals bei einer Datenübertragung}

\newglossaryacronym{lora}{LoRa}{Long Range}{proprietärer Funkstandard mit Fokus auf hoher Reichweite bei geringem Energiebedarf}

\newglossaryacronym{sf}{SF}{Spreading Factor}{steuert Übertragunsrate bei \acrshort{LoRa}-Übertragung, höhere Reichweite und längere Übertragungsdauer bei Erhöhung}

\newglossaryacronym{srd}{SRD}{Short Range Device Band}{lizenzfreies Frequenzband für Übertragungen über kurze Reichweiten bzw. geringer Leistung}

\newglossaryacronym{ism}{ISM}{{Industrial, Scientific and Medical Band}}{lizenzfrei nutzbare Frequenzbereiche}

\newglossaryacronym{lorawan}{LoRaWAN}{Long Range Wide Area Network}{auf Energiesparsamkeit bei der Datenübertragung ausgelegtes Weitverkehrsnetzwerk unter Nutzung des Funkstandards \acrshort{LoRa}}

\newglossaryacronym{ttn}{TTN}{The Things Network}{weltweites, offenes und communitybasiertes Weitverkehrsnetzwerk auf Basis von \acrshort{LoRaWAN}}

\newglossaryacronym{node}{Node}{Endgerät}{sendet und empfängt Daten im \acrshort{LoRaWAN} und verarbeitet diese (z.\,B. Füllstandssensor mit Alarm und fernsteuerbarem Ventil)}

\newglossaryacronym{abp}{ABP}{Activation By Personalization}{Aktivierung \acrshort{Node} im \acrshort{LoRaWAN} mit vorberechneten Sitzungsschlüsseln}

\newglossaryacronym{otaa}{OTAA}{Over-The-Air-Activation}{Aktivierung \acrshort{Node} im \acrshort{LoRaWAN} über Join-Prozess mit dynamischer Berechnung Sitzungsschlüssel}

\newglossaryacronym{crc}{CRC}{Cyclic Redundancy Check}{Prüfwert für Fehlererkennung oder -korrektur}

\newglossaryentry{Gateway}
{
  name=Gateway,
  description={Schnittstelle zwischen \acrshort{LoRa}-Kommunikation und Netzwerkserver im \acrshort{LoRaWAN}}
}

\newglossaryentry{Uplink}
{
  name=Uplink,
  description={Nachricht von \acrshort{Node} via \gls{Gateway} an \acrshort{LoRaWAN}-Anwendung}
}

\newglossaryentry{Downlink}
{
  name=Downlink,
  description={Nachricht von \acrshort{LoRaWAN}-Anwendung via \gls{Gateway} an \acrshort{Node}}
}

\newglossaryentry{Airtime}
{
  name=Airtime,
  description={Dauer, welche zum Senden einer Nachricht benötigt wird}
}

\newglossaryentry{Chirp}
{
  name=Chirp,
  description={Signal steigender oder fallender Frequenz mit konstanter Amplitude}
}

\newglossaryentry{Modulation}
{
  name=Chirp Spread Spectrum,
  description={Modulationsverfahren unter Verwendung von \glspl{Chirp}}
}

\newglossaryentry{Netserver}
{
  name=Network Server,
  description={TODO }
}
\newglossaryentry{Appserver}
{
  name=Application Server,
  description={TODO }
}

\newglossaryentry{Payload}
{
  name=Payload,
  description={Nutzdaten  zwischen verschiedenen Objekten, das heißt ohne Steuer- und Protokollinformationen}
}

\newglossaryentry{Payload-Formatter}
{
  name=Payload-Formatter,
  description={Funktion zur Dekodierung der Bytes einer \gls{Payload} zu einem definierten Datenschema}
}
