\chapter{Zusammenfassung}


LoRa stellt als Grundlage eine solide Übertragungstechnik für die Kommunikation über hohe Reichweiten mit niedrigem Energiebedarf dar.
Die Chirp Spread Spectrum Modulation ist ideal, um Signale mit geringer Leistung auch bei starkem Rauschen fehlerfrei übertragen zu können.
Über verschiedene Spreading Factors lassen sich Energiebedarf und Reichweite gezielt steuern.

LoRaWAN bringt per LoRa kommunizierende Geräte zu einem einfach nutzbaren Netzwerk zusammen.
Durch Verschlüsselungsmechanismen ist die Vertraulichkeit der Kommunikation gewährleistet.
Durch verschiedenen Geräteklassen lässt sich ein Kompromiss zwischen Energiebedarf und Latenz der Kommunikation bilden und Verfahren wie Adaptive Data Rate ermöglichen eine automatische Optimierung von Energiebedarf und Reichweite.

LoRaWAN bildet insbesondere für Bereiche, in denen niedrige Datenraten und höhere Reaktionszeiten kein Problem sind, eine gute Grundlage für Internet of Things.
Dies ist zum Beispiel für die Erfassung von Sensorwerten von räumlich weit verteilten Sensoren der Fall.