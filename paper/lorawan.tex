\chapter{LoRaWAN}
Mittels LoRaWAN können über LoRa kommunizierende Endgeräte zu einem Netzwerk (WAN) zusammengeschlossen werden.
LoRaWAN gibt dabei die Systemarchitektur und die zur Kommunikation genutzten Protokolle vor.


\section{Architektur}
Der grundlegende Architektur von LoRaWAN ist in \autoref{img.lorawan-architecture} dargestellt.
Die Komponenten sind von links nach rechts mit steigender Abstraktion von der eigentlichen LoRa-Kommunikation angeordnet.
Eine Nachricht vom Endgerät zur Applikation hin wird als \gls{Uplink}, eine Nachricht von der Applikation zum Endgerät hin als \gls{Downlink} bezeichnet.
\cite{lorawanarchitecture}
\image{lorawan-architecture}{img.lorawan-architecture}{\textwidth}{Systemarchitektur bei LoRaWAN}{Systemarchitektur bei LoRaWAN}{bearbeitet aus Zerynth srl, https://www.zerynth.com/wp-content/uploads/2017/05/lorawan-architecture.jpg}


\subsection{Endgerät}
TODO 

\subsubsection{Uplink}
TODO

\subsubsection{Downlink}
TODO


\subsection{Gateway}
Ein Gateway bildet die Schnittstelle zwischen der Funkkommunikation per LoRa und den restlichen Komponenten des Netzwerks.
LoRa zur Datenübertragung wird somit lediglich von Endgeräten und Gateways genutzt.
TODO
In einem LoRaWAN können beliebig viele Gateways genutzt werden.
%Durch hinzufügen von Gateways kann der Durchsatz der Kommunikation zu den Endgeräten erhöht werden.

\subsubsection{Uplink}
Ein Gateway empfängt die von den Endgeräten über LoRa gesendeten Nachrichten.
Außerdem werden Metadaten wie Signalstärke und \gls{SNR} der empfangenen Signale bestimmt.
Die Nutzdaten der LoRa-Nachricht und die Metadaten werden vom Gateway an den Netzwerkserver weitergeleitet, üblicherweise per Internet.

\subsubsection{Downlink}
Ebenso kann das Gateway Daten vom Netzwerkserver empfangen.
Diese werden per LoRa gesendet und können von Endgeräten in Reichweite des Gateways empfangen werden.


\subsection{Netzwerkserver}
Der Netzwerkserver bildet im LoRaWAN somit das zentrale Element für das Routing der empfangenen und zu sendenden Nachrichten. Es gibt genau einen Netzwerkserver im LoRaWAN.

\subsubsection{Uplink}
Der Netzwerkserver verarbeitet die von den Gateways empfangenen Nachrichten.
Wie in \autoref{img.lorawan-architecture} unten links dargestellt, kann die gesendete Nachricht eines Endgeräts von mehreren Gateways gleichzeitig empfangen werden.
Dabei leitet jedes Gateway die Nachricht an den Netzwerkserver weiter.
Durch Vergleich des (teilw. verschlüsselten) Dateninhalts der  Nachricht können mehrfach empfangene Nachrichten im Netzwerkserver zu einer einzigen Nachricht zusammengefasst werden.
Dies wird auch als Deduplizierung bezeichnet.

Außerdem wird für jedes Endgerät gespeichert, über welches Gateway die Signalqualität der empfangenen Daten am besten ist.

Sofern das Endgerät, von dem die Nachricht empfangen wurde, im Netzwerk aktiviert ist (siehe \autoref{sec.lorawan.activation}), kann die Netzwerkverschlüsselung der Nachricht entfernt werden.
Die entschlüsselten Daten werden an den Applikationsserver weitergeleitet, der für dieses Endgerät festgelegt wurde.

\subsubsection{Downlink}
Ebenso kann der Netzwerkserver Downlinkpakete vom Applikationsserver empfangen.
Diese werden zunächst in einer Warteschlange gespeichert (Downlink-Queue).
Abhängig von der Geräteklasse des adressierten Endgeräts erfolgt die weitere Verarbeitung direkt oder durch bestimmte Auslöser (siehe \autoref{sec.lorawan.classes}).
In der Weiterverarbeitung werden die Paketdaten zunächst der Netzwerkverschlüsselung unterzogen.
Außerdem wird für das adressierte Endgerät das Gateway bestimmt, welches aktuell wahrscheinlich die besten Übertragungsbedingungen zum Endgerät hat.
Dies wird anhand der gespeicherten Daten über die Signalqualität vorher empfangener Pakete durchgeführt.
Anschließend werden die verschlüsselten Daten an das ermittelte Gateway weitergeleitet.
Ein Paket im Downlink wird also immer nur über genau ein Gateway gesendet.


\subsection{Applikationsserver}
Im LoRaWAN kann es beliebig viele Applikationsserver geben.
Diese dienen der Ent- bzw. Verschlüsselung der Applikationsdaten und sind die Schnittstelle des LoRaWAN zu eigentlichen Anwendungen.

\subsubsection{Uplink}
Ein Applikationsserver empfängt vom Netzwerkserver gesendete Daten.
Dabei wird zunächst die Applikationsverschlüsselung entfernt.
Anschließend sind die eigentlichen vom Endgerät gesendeten Daten (\gls{Payload}) im Klartext verfügbar und können beliebig verarbeitet werden.
Die Metadaten der LoRa-Kommunikation, welche durch die Gateways erfasst wurden (z.\,B. \gls{SNR}), sind ebenso Bestandteil der im Applikationsserver verfügbaren Daten.
Die weitere Verarbeitung der Uplinks ist allerdings nicht mehr Bestandteil der Architektur von LoRaWAN.

\subsubsection{Downlink}
Im Applikationsserver können außerdem Downlinks gestartet werden, also Nachrichten zum Endgerät gesendet werden.
Dabei werden die Daten der Applikationsverschlüsselung unterzogen und das Paket an den Netzwerkserver gesendet.


\subsection{Join-Server}
TODO


\section{Verschlüsselung}
TODO


\section{Registrierung und Aktivierung}\label{sec.lorawan.activation}
TODO
siehe \autoref{tab.lorawan-activation} und \autoref{img.lorawan-OTAA}
\begin{table}[htbp]
\begin{tabular}{l|l|l}
 & \textbf{Activation by Personalization (ABP)} & \textbf{Over-the-Air Activation (OTAA)} \\ \hline
\textbf{Sitzung} & eine statische Sitzung & dynamisch, beliebig oft neu \\ \hline
\textbf{Schlüssel} & \begin{tabular}[c]{@{}l@{}}Sitzungsschlüssel fest\\ einprogrammiert\end{tabular} & \begin{tabular}[c]{@{}l@{}}Aushandlung für Sitzung,\\ Ableitung aus "`Master-Key"'\end{tabular} \\ \hline
\textbf{Beitritt} & Kommunikation direkt möglich & Beitrittsprozess nötig
\end{tabular}
\caption{Vergleich der Methoden für Aktivierung von Endgeräten}
\label{tab.lorawan-activation}
\end{table}

\image{lorawan-OTAA}{img.lorawan-OTAA}{\textwidth}{Beitrittsprozess}{Beitrittsprozess}{eigene Darstellung}



\section{Geräteklassen}\label{sec.lorawan.classes}
Endgeräten im LoRaWAN wird eine Geräteklasse zugewiesen.
Diese ändert die Zeitpunkte, zu denen das Gerät empfangsbereit ist, also einen Downlink vom Gateway empfangen kann.
TODO siehe \autoref{img.lorawan-classes}

\image{lorawan-classes}{img.lorawan-classes}{\textwidth}{Empfangsfenster der Geräteklassen bei LoRaWAN}{Geräteklassen bei LoRaWAN}{https://witekio.com/wp-content/uploads/2018/01/Lora-wan-class.png}




\section{ADR}
TODO
\image{lorawan-ADR}{img.lorawan-ADR}{\textwidth}{Ablauf ADR}{Ablauf ADR}{eigene Darstellung}


Frequenzen werden durchgewechselt, um nicht auf einer Frequenz mit vielen Störungen hängen zu bleiben

