\chapter{Anwendungsbeispiele}
Im folgenden Kapitel werden kurze Beispiele zu Anwendungen von LoRa und LoRaWAN vorgestellt.



\section{LoRa}
Für den LoRa-fähigen Mikrocontroller Cubecell\footnote{\url{https://heltec.org/project/htcc-ab02s/}} wurde eine Software entwickelt, mit der die Kommunikation über LoRa getestet werden kann.
Ein Gerät wird dabei als Client genutzt und sendet die per serieller Verbindung übergebenen Daten.
Anschließend versucht das Gerät eine potentielle Antwort zu empfangen.
Ein weiteres Gerät wird als Server genutzt, welcher ständig auf ankommende Pakete wartet.
Wird ein Paket empfangen, so werden die in den Daten enthaltenen Buchstaben zu Großbuchstaben transformiert und per LoRa gesendet.
Durch Client und Server werden die gesendeten und empfangenen Daten sowie die Signalstärke und das \gls{SNR} des empfangenen Pakets jeweils auf dem Bildschirm des Mikrocontrollers dargestellt.

Das Protokoll des Clients ist in \autoref{img.lora-demo-console} dargestellt.
TX bezeichnet dabei die jeweils gesendeten Daten und RX die empfangenen.
In der Demonstration wurde zunächst über wenige Zentimeter Abstand und mit angebrachter Antenne kommuniziert (Nachricht "`Hallo Welt"' und "`Moin"').
Durch Entfernen der Antennen konnte gezeigt werden, dass die Signalstärke  sinkt (Nachricht "`1. Antenne weg"' und "`..."' mit nur einer Antenne, Nachricht "`Antennen weg"' komplett ohne Antennen).
Außerdem wurde die Signalübertragung über mehrere Etagen in den Keller getestet.
Dabei wurde festgestellt, dass ohne Antenne keine Kommunikation möglich ist (erste Nachricht "`Keller?"').
Mit angebrachter Antenne kann problemlos durch mehrere Wände kommuniziert werden, die Signalstärke und das \gls{SNR} sinken allerdings im Vergleich zur Übertragung in über wenige Zentimeter (zweite Nachricht "`Keller?"').
\image{lora-demo-console}{img.lora-demo-console}{0.8\textwidth}{Protokoll des Clients bei Tests der LoRa-Signalübertragung}{Protokoll des Clients bei Tests der LoRa-Signalübertragung}{eigene Darstellung}

Der Quellcode der Software ist auf Github verfügbar.\footnote{\url{https://github.com/Salaml/sensor-networks/tree/main/code}}
Eine Live-Demonstration ist auf Youtube abrufbar.\footnote{\url{https://www.youtube.com/watch?v=YyFAu_R3ZoA}}



\section{LoRaWAN}
In der HTW Dresden wurde ein LoRaWAN-Gateway installiert und in das freie LoRaWAN \gls{TTN} eingebunden.
Im folgenden wird kurz auf einige durch das Gateway bereitgestellte bzw. erfasste Statistiken eingegangen.

In \autoref{img.gateway-duty-cycle} ist der Duty Cycle des Gateways nach Kanal aufgeschlüsselt dargestellt.
Es ist erkennbar, dass einige Kanäle häufiger verwendet werden als andere.
Außerdem wird deutlich, dass LoRa-Übertragungen nicht dauerhaft stattfinden sondern aufgrund des gesetzlichen begrenzten Duty Cycles nur zu sporadischen Zeitpunkten.
\image{gateway-duty-cycle}{img.gateway-duty-cycle}{0.6\textwidth}{Duty Cycle des LoRaWAN-Gateways der HTW Dresden nach Kanal}{Duty Cycle des LoRaWAN-Gateways der HTW Dresden nach Kanal}{eigene Darstellung}

In \autoref{img.gateway-range} ist die Reichweite des Gateways erkennbar.
\image{gateway-range}{img.gateway-range}{\textwidth}{kartierte Signalstärke des LoRaWAN-Gateways der HTW Dresden}{kartierte Signalstärke des LoRaWAN-Gateways der HTW Dresden}{TTN-Mapper, https://ttnmapper.org/heatmap/private/?gateway=htw-dresden-ttn-gw1\&network=NS\_TTS\_V3://ttn@000013}
Die Punkte sind nach der Signalstärke gefärbt, mit der eine Nachricht vom angegebenen Punkt das Gateway erreicht hat.
Die Daten werden erfasst, indem ein Node periodisch (beliebige) Daten ins LoRaWAN sendet während er mit dem Kartierenden in der Umwelt bewegt wird.
Die empfangenen Daten werden durch das Mobiltelefon des Kartierenden vom Applikationsserver über eine MQTT-Schnittstelle abgerufen.
Das Mobiltelefon kombiniert die aus den Metadaten verfügbare Signalstärke zusammen mit der aktuellen Position und sendet diese dann an den Server des TTN-Mapper.
Dort werden die Daten aufbereitet und entsprechend auf der Karte dargestellt.
Dieser Workflow demonstriert perfekt den Datenfluss durch ein LoRaWAN.

