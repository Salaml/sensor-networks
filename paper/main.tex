\documentclass[%
	fontsize=12pt,%     Schrfitgroesse 12pt
	paper=a4,%          Seitengroesse A4
	toc=listof,%        Abbildungs-, Tabellen- sowie ...
	toc=bibliography,%  ... Literaturverzeichnis als Eintrag im Inhaltsverzeichnis
]{scrreprt}% KOMA-Script-Report

\usepackage[utf8]{inputenc}% Dateien UTF-8 kodiert
\usepackage[T1]{fontenc}%    T1-Schriftkodierung
\usepackage{lmodern}%        Latin-Modern-Schriftart
\usepackage[ngerman]{babel}% neue deutsche Rechtschreibung

\usepackage[top=30mm, bottom=30mm, inner=25mm, outer=25mm]{geometry}% Seitenrand

\usepackage[onehalfspacing]{setspace}% Zeilenabstand 1,5-fach

\usepackage{graphicx} % Bilder
\usepackage[%
	font=footnotesize,% Schriftgroesse wie bei Fussnoten
	center,%            zentrierte Ausrichtung
	format=plain%       kein Einzug bei merhzeiligem Text (z. B. bei '\\Quelle: xy')
]{caption}% Beschriftung fuer Abbildungen/Tabellen

\graphicspath{{../images}} % Standardpfad fuer Bilder
\DeclareGraphicsExtensions{.pdf,.png,.jpg} % bevorzuge pdf-Dateien vor den anderen

\newcommand{\image}[6]{
	\begin{figure}[htbp]
		\centering
		\includegraphics[width=#3]{#1}
		\caption[#5]{#4 \\
			\IfStrEq{#6}{}{}{Quelle: {#6}}
		}
		\label{#2}
	\end{figure}
}

\usepackage{xstring}  %  ifcase

\usepackage{csquotes}
\usepackage[
	backend=biber,%
	sortlocale=de_DE,%
	bibencoding=utf8,%
	style=alphabetic,%
]{biblatex}	% Literatur/Quellen


\input{htwcd/htwcd.sty}% Corporate Design HTW anwenden

% ======================================================
% Informationen für PDF-Dokument:
% ======================================================
\usepackage[%bookmarks, colorlinks=true, linkcolor=black, citecolor=black, urlcolor=black,%
	pdfauthor={Salaml},%
	pdftitle={LoRa / LoRaWAN},%
	pdfsubject={Sensornetze},%
	breaklinks=true%
]{hyperref}% Verlinkungen von Inhaltsverzeichnis, Abbildungen etc. innerhalb des Dokuments
% ======================================================

\addbibresource{bibliography.bib}% Bibliographie einbinden
\usepackage[
	nonumberlist,%  keine Seitenzahlen anzeigen
	acronym,%       ein Abkürzungsverzeichnis erstellen
	toc,%           Einträge im Inhaltsverzeichnis
	nopostdot,%     kein Punkt am Ende der Beschreibung
	automake%
]{glossaries}% Glossar

\newglossary[slg]{symbolslist}{syi}{syg}{Symbolverzeichnis}% eigenes Symbolverzeichnis erstellen

\renewcommand{\glsdescwidth}{0.7\textwidth}% Breite bei Typ long erhöhen

% Befehl für Eintrag, der sowohl im Glossar als auch im Abkürzungsverzeichnis auftaucht
\newcommand{\newglossaryacronym}[4]{
	\newglossaryentry{#1}{
		name=\glslink{#2}{#3},
		text=#3,
		description={#4},
		sort={#3}% nach Text sortieren
	}
	\newacronym[description={\glslink{#1}{#3}}]{#2}{#2}{#3}
}

\makeglossaries% Glossar erzeugen

% Symbole
\newglossaryentry{sym.Rc}{
	name=$R_c$,
	description={Chiprate der Signalübertragung},
	sort=rc, type=symbolslist
}

\newglossaryentry{sym.B}{
	name=$B$,
	description={Bandbreite der Signalübertragung},
	sort=b, type=symbolslist
}

\newglossaryentry{sym.fmax}{
	name=$f_{max}$,
	description={maximale Frequenz des \glspl{Chirp} bei der Signalübertragung},
	sort=fmax, type=symbolslist
}

\newglossaryentry{sym.fmin}{
	name=$f_{min}$,
	description={minimale Frequenz des \glspl{Chirp} bei der Signalübertragung},
	sort=fmin, type=symbolslist
}

\newglossaryentry{sym.Rs}{
	name=$R_s$,
	description={Symbolrate der Signalübertragung},
	sort=rs, type=symbolslist
}

\newglossaryentry{sym.SF}{
	name=$SF$,
	description={\acrlong{SF} der Signalübertragung},
	sort=sf, type=symbolslist
}

\newglossaryentry{sym.Rb}{
	name=$R_b$,
	description={Bitrate der Signalübertragung},
	sort=rb, type=symbolslist
}

\newglossaryentry{sym.CR}{
	name=$CR$,
	description={Code Rate für Fehlerkorrektur},
	sort=cr, type=symbolslist
}

\newglossaryentry{sym.Rbnet}{
	name=$R_{b,netto}$,
	description={Nettobitrate der Signalübertragung},
	sort=rbnet, type=symbolslist
}

% Beispiel: Eintrag, der sowohl im Glossar als auch im Abkürzungsverzeichnis auftaucht:
\newglossaryacronym{ID für Verweis auf Glossar}{Abkürzung / ID für Verweis Abk.-Verzeichnis}{langer Name}{Beschreibung im Glossar}

\newglossaryacronym{snr}{SNR}{Signal-Rausch-Verhältnis}{Maß für die Qualität des Nutzsignals bei einer Datenübertragung}

\newglossaryacronym{lora}{LoRa}{Long Range}{proprietärer Funkstandard mit Fokus auf hoher Reichweite bei geringem Energiebedarf}

\newglossaryacronym{sf}{SF}{Spreading Factor}{steuert Übertragunsrate und \gls{Airtime} bei \acrshort{LoRa}-Übertragung}

\newglossaryacronym{srd}{SRD}{Short Range Device Band}{lizenzfreies Frequenzband für Übertragungen über kurze Reichweiten bzw. geringer Leistung}

\newglossaryacronym{ism}{ISM}{{Industrial, Scientific and Medical Band}}{lizenzfrei nutzbare Frequenzbereiche}

\newglossaryacronym{lorawan}{LoRaWAN}{Long Range Wide Area Network}{auf Energiesparsamkeit bei der Datenübertragung ausgelegtes Weitverkehrsnetzwerk unter Nutzung des Funkstandards \acrshort{LoRa}}

\newglossaryacronym{ttn}{TTN}{The Things Network}{weltweites, offenes und communitybasiertes \acrshort{LoRaWAN}}

\newglossaryacronym{abp}{ABP}{Activation by Personalization}{Aktivierung des \gls{Node} im \acrshort{LoRaWAN} mit vorberechneten Sitzungsschlüsseln}

\newglossaryacronym{otaa}{OTAA}{Over-The-Air-Activation}{Aktivierung des \gls{Node} im \acrshort{LoRaWAN} über Join-Prozess mit dynamischer Berechnung Sitzungsschlüssel}

\newglossaryacronym{crc}{CRC}{Cyclic Redundancy Check}{Prüfwert für Fehlererkennung oder -korrektur}

\newglossaryentry{Gateway}{
	name=Gateway,
	description={Schnittstelle zwischen \acrshort{LoRa}-Kommunikation und Netzwerkserver im \acrshort{LoRaWAN}}
}

\newglossaryentry{Node}{
	name=Node,
	description={Endgerät im \acrshort{LoRaWAN}, kann per \acrshort{LoRa} Daten senden und empfangen}
}

\newglossaryentry{Uplink}{
	name=Uplink,
	description={Nachricht von \gls{Node} via \gls{Gateway} an \acrshort{LoRaWAN}-Anwendung}
}

\newglossaryentry{Downlink}{
	name=Downlink,
	description={Nachricht von \acrshort{LoRaWAN}-Anwendung via \gls{Gateway} an \gls{Node}}
}

\newglossaryentry{Airtime}{
	name=Airtime,
	description={Dauer, welche zum Senden einer Nachricht benötigt wird}
}

\newglossaryentry{Chirp}{
	name=Chirp,
	description={Signal steigender oder fallender Frequenz mit konstanter Amplitude}
}

\newglossaryentry{Modulation}{
	name=Chirp Spread Spectrum,
	description={Modulationsverfahren unter Verwendung von \glspl{Chirp}}
}

\newglossaryentry{Netserver}{
	name=Netzwerkserver,
	description={zentrales Element für Routing von Nachrichten zwischen \glspl{Gateway} und \gls{Appserver} im \gls{LoRaWAN}}
}

\newglossaryentry{Appserver}{
	name=Applikationsserver,
	description={Schnittstelle zwischen \gls{LoRaWAN} und Anwendungen}
}

\newglossaryentry{Joinserver}{
	name=Join-Server,
	description={Server für Schlüsselverwaltung und -verteilung im \gls{LoRaWAN}}
}

\newglossaryentry{Payload}{
	name=Payload,
	description={Nutzdaten zwischen verschiedenen Objekten, das heißt ohne Steuer- und Protokollinformationen}
}% Glossar und Akronyme einbinden

% ======================================================
% Informationen für Titelseite:
% ======================================================
\subject{Sensornetze}
\title{LoRa / LoRaWAN}
\author{{Salaml}}
\date{Wintersemester 2021/2022}

% ======================================================
% Inhalt
% ======================================================

\begin{document}
\pagenumbering{Roman} % Seitennummerierung Verzeichnisse mit großen römischen Ziffern 
\pagestyle{empty} % keine Kopf- oder Fußzeilen auf den ersten Seiten
\maketitle % Titelseite

\pagestyle{plain} % normale Kopf- und Fußzeilen für den Rest

\begin{singlespacing}
	\tableofcontents % Inhaltsverzeichnis

	\printglossary[type=symbolslist,style=longheader]% Symbole
	\printglossary[type=\acronymtype,style=longheader,title=Abkürzungsverzeichnis]% Abkürzungen
	\printglossary[style=altlist,title=Glossar]% Glossar

	\listoffigures % Abbildungsverzeichnis
	\listoftables % Tabellenverzeichnis
\end{singlespacing}
\clearpage

\pagenumbering{arabic} % Seitennummerierung Inhalt mit arabischen Ziffern

\chapter{Einführung}
Das Internet of Things (IoT) ist nicht nur in der IT-Branche ein aktuelles Thema sondern dringt auch immer mehr in die Welt von Otto Normalverbraucher vor.
Das Einstellen der Heizkörpertemperaturen, Schalten von Steckdosen oder Protokollieren von Messwerten wie Temperatur oder Feinstaubgehalt der Luft über das Internet ist für viele Menschen bereits Normalität.
Das \gls{TTN} bildet einen für jeden nahezu frei nutzbaren Zugang zum Internet of Things mit großer räumlicher Abdeckung.

Grundlage des TTN bildet das Ökosystem aus LoRa und LoRaWAN.
LoRa ist dabei eine Funktechnologie, die zur Übertragung von Daten über große Distanzen mit wenig Energieaufwand verwendet wird. 
Mittels LoRaWAN können Geräte, die LoRa beherrschen, in ein Netzwerk eingebunden werden und damit Teil des Internet of Things werden.
Kernelement von LoRa und LoRaWAN ist dabei ein möglichst niedriger Energiebedarf der Endgeräte.

\image{lorawan-architecture-simple}{img.arch-simple}{0.9\textwidth}{Architektur von LoRaWAN (vereinfacht)}{Architektur von LoRaWAN (vereinfacht)}{bearbeitet aus Zerynth srl, https://www.zerynth.com/wp-content/uploads/2017/05/lorawan-architecture.jpg}
Anhand von \autoref{img.arch-simple} soll die Abgrenzung von LoRa und LoRaWAN verdeutlicht werden.
Über LoRa wird lediglich zwischen den Endgeräten und den Gateways kommuniziert, den Schnittstellen zum weiteren Netzwerk.
Jegliche weitere Kommunikation wird über das Internet abgewickelt.
Die gesamte Abbildung stellt das LoRaWAN dar, LoRa ist nur die Funktechnologie die auf der untersten Netzebene genutzt wird.

In den folgenden Kapiteln wird zunächst näher auf LoRa als Grundlage der Kommunikation eingegangen, anschließend wird die Vernetzung der Geräte zum LoRaWAN erläutert.

\chapter{LoRa}\label{chp.lora}



\section{Grundlagen}\label{sec.lora.basics}
LoRa ist ein proprietäres Übertragungsverfahren per Funk.
Es wurde vom Unternehmen Semtech entwickelt, um Kommunikation mit niedrigem Energiebedarf über große Distanzen zu ermöglichen.
Im Vergleich zu anderen Funktechnologien wie Bluetooth oder Mobilfunk, ist die "`Bandbreite"' der Datenübertragung (Datenrate) bei LoRa nur gering.
Dagegen ist die erzielbare Reichweite größer, wie \autoref{img.bandwidth-range} zeigt.
\cite{semtech}

\image{bandwidth-vs-range}{img.bandwidth-range}{\textwidth}{Bandbreite (Datenrate) und Reichweite von LoRa im Vergleich zu anderen Funktechnologien}{Bandbreite und Reichweite von LoRa im Vergleich zu anderen Funktechnologien}{The Things Industries, https://www.thethingsnetwork.org/docs/lorawan/what-is-lorawan/bandwidth-vs-range.png}

Für LoRa werden ortsabhängig Frequenzbereiche aus lizenzfreien Bändern genutzt (\gls{ISM}, \gls{SRD}).
In Europa kommt das SRD-Band um 868\,MHz zum Einsatz.\cite{loraparameter}
Obwohl es sich hierbei um ein freies Frequenzband handelt, sind dennoch einige Einschränkungen zu beachten:\cite{lorabasics}
\begin{itemize}\singlespacing\setlength\itemsep{-0.2em}
\item Sendeleistung \gls{Uplink} begrenzt auf 25\,mW (14\,dBm)
\item Sendeleistung \gls{Downlink} begrenzt auf 500\,mW (27\,dBm)
\item Duty Cycle Senden pro Gerät begrenzt auf kanalabhängig 0,1\% bzw. 1\%
\end{itemize}
Diese Grenzen sind gesetzlich festgelegt, um eine Blockade des Bandes unter anderem durch zu häufiges Senden zu verhindern.
Damit ist jedem eine faire Verwendung dieses Bandes unter Beachtung der Grenzen möglich.

Um insbesondere die Begrenzung der Sendezeit inzuhalten zu können, sollte eine Minimierung der Sendezeit (\gls{Airtime}) erfolgen.



\section{Modulation Chirp Spread Spectrum }\label{sec.lora.modulation}
Per LoRa gesendete Signale werden mittels des Verfahrens \gls{Modulation} moduliert.
Grundlage dieser Modulation sind sogenannte \glspl{Chirp}.
Eine Anpassung der Datenrate ist über den sogenannten \gls{SF} möglich.


\subsection{Chirps}\label{sec.lora.modulation.chirp}
Ein Chirp ist ein Impuls mit konstanter Amplitude und sich ändernder Frequenz.
Bei steigender Frequenz wird der Chirp als Upchirp bezeichnet, bei fallender als Downchirp.
In \autoref{img.chirp-linear} ist ein Upchirp mit linearem Anstieg der Frequenz veranschaulicht.
\image{chirp-linear}{img.chirp-linear}{0.7\textwidth}{Upchirp mit linearem Frequenzanstieg}{Upchirp mit linearem Frequenzanstieg}{Georg-Johann, CC BY-SA 3.0, via Wikimedia Commons, https://commons.wikimedia.org/wiki/File:Linear-chirp.svg}

Die Änderung der Frequenz der Chirps erfolgt nicht kontinuierlich, sondern zu diskreten Zeitpunkten.
Ein einzelner Zustand heißt dabei Chip, die Anzahl an Frequenzänderungen pro Zeiteinheit wird als Chiprate $R_c$ bezeichnet.
Bei LoRa entspricht dies der von den Chirps genutzten Bandbreite $B$ (üblicherweise 125\,kHz).
\cite[S. 10]{loramodulation}
\begin{equation}
	R_c = B = 125\,\mathrm{kHz}
	\label{eq.chip-rate}
\end{equation}


\subsection{Symbole}
Zur Kodierung von Daten wird ein Chirp zeitlich verschoben und bildet damit ein sogenanntes Symbol.
Die Startfrequenz des Symbols legt dabei die kodierten Daten fest.
Nachdem die Frequenz des Chirps innerhalb des Symbols die Grenze der Bandbreite erreicht hat, wird der Chirp mit der Frequenz der anderen Bereichsgrenze fortgesetzt (Upchirp: Sprung von $f_{max}$ zu $f_{min}$, Downchirp: Sprung von $f_{min}$ zu $f_{max}$).
\cite{lorachirps}

\image{lora-symbols}{img.lora-symbols}{\textwidth}{Spektrum verschiedener LoRa-Symbole}{Spektrum verschiedener LoRa-Symbole}{bearbeitet aus Sakshama Ghoslya, https://www.sghoslya.com/p/lora\_9.html}
In \autoref{img.lora-symbols} ist der Frequenzgang verschiedener Symbole anhand des Spektrums veranschaulicht.
Innerhalb der Symbole ist jeweils der Sprung von der maximalen Frequenz $f_{max}$ zur minimalen Frequenz $f_{min}$ und damit der Lauf über den kompletten Frequenzbereich erkennbar.
Innerhalb eines Symbols wird dementsprechend immer die komplette Bandbreite genutzt, lediglich die Startfrequenz unterscheidet sich nach den kodierten Daten.
Aufgrund dieser Verteilung des Signals über das Spektrum wird das Modulationsverfahren Chirp Spread Spectrum genannt.

Die Anzahl der Symbole pro Zeiteinheit wird als Symbolrate $R_s$ bezeichnet.


\subsection{Spreading Factor}
Die Dauer und der Informationsgehalt der Symbole kann über denn sogenannten Spreading Factor $SF$ eingestellt werden.
Dieser gibt die Anzahl Bits an, die in einem Symbol kodiert werden können.
Für die Symbolrate gilt die in \autoref{eq.symbol-rate} dargestellte Beziehung.
\cite[S. 10]{loramodulation}
\begin{equation}
	R_s = \frac{B}{2^{SF}} = \frac{R_c}{2^{SF}}
	\label{eq.symbol-rate}
\end{equation}
Bei Erhöhung des Spreading Factors um 1 halbiert sich damit die Symbolrate, die Dauer der Symbole verdoppelt sich.
In \autoref{img.lora-spreading-factors} ist die Verdopplung der Zeitdauer bei Erhöhung des Spreading Factors deutlich erkennbar.
\image{lora-spreading-factors}{img.lora-spreading-factors}{\textwidth}{Spektrum verschiedener Spreading Factors}{Spektrum verschiedener Spreading Factors}{bearbeitet aus Sakshama Ghoslya, https://www.sghoslya.com/p/lora-is-chirp-spread-spectrum.html}

% TODO Beziehung zu Chirprate
% TODO coding rate -> 4 Bits + 1 bis 4 Bits
% R_b = SF * BW / 2^SF
% R_b = SF * (4 / 4+CR) / (2^SF / BW)

LoRa-Empfänger weisen eine bessere Empfindlichkeit gegenüber Signalen mit höherem Spreading Factor auf, wie in \autoref{tab.spreading-factors} dargestellt.
\begin{table}[htbp]
\centering
\begin{tabular}{@{}lcr@{}}
\textbf{Spreading Factor} & \textbf{Empfindlichkeit} & \textbf{Airtime} \\
SF7  & -123,0\,dBm &  41\,ms \\
SF8  & -126,0\,dBm &  72\,ms \\
SF9  & -129,0\,dBm & 144\,ms \\
SF10 & -132,0\,dBm & 288\,ms \\
SF11 & -134,5\,dBm & 577\,ms \\
SF12 & -137,0\,dBm & 991\,ms \\
\end{tabular}
\caption[Empfindlichkeit und Airtime verschiedener Spreading Factors]{Empfindlichkeit und Airtime verschiedener Spreading Factors\\Quelle: Semtech Corporation, https://lora-developers.semtech.com/documentation/tech-papers-and-guides/understanding-adr}
\label{tab.spreading-factors}
\end{table}
Eine Erhöhung des Spreading Factors führt damit zu folgenden Effekten:
\begin{itemize}\singlespacing\setlength\itemsep{-0.2em}
\item Airtime Sender erhöht
\item Energiebedarf Sender erhöht
\item Empfindlichkeit Empfänger verbessert
\item Übertragungsreichweite erhöht
\end{itemize}
Für diese Werte gelten jedoch unterschiedliche Ziele: Airtime und Energiebedarf sollten minimiert werden, Empfindlichkeit und Reichweite maximiert werden.\cite{loralimits}
Um einen bestmöglichen Kompromiss zwischen diesen Vorgaben zu schaffen, implementiert LoRaWAN die in \autoref{sec.lorawan.adr} beschriebene automatische Anpassung des Spreading Factors.


\subsection{Orthogonalität}
TODO


\section{LoRa-Nachricht}
Eine per LoRa gesendete Nachricht besteht aus den folgenden Elementen:\cite{loralimits}
\begin{itemize}\singlespacing\setlength\itemsep{-0.2em}
\item Präambel
\item Header + \gls{CRC} für Header
\item Nutzdaten
\item \gls{CRC}
\end{itemize}

Die Präambel dient zur Synchronisation des Empfängers auf die gesendeten Daten.
Anhand der Präambel wird außerdem der Spreading Factor erkannt, mit welcher die Daten gesendet wurden.

Der Header enthält unter anderem Informationen zur Länge der Nutzdaten.

Am Ende der Nachricht folgt optional ein CRC-Wert zu Fehlerdetektion bzw. -korrektur.
\chapter{LoRaWAN}
Mittels LoRaWAN können über LoRa kommunizierende Endgeräte zu einem Netzwerk (WAN) zusammengeschlossen werden.
LoRaWAN gibt dabei die Systemarchitektur und die zur Kommunikation genutzten Protokolle vor.



\section{Architektur}
Der grundlegende Architektur von LoRaWAN ist in \autoref{img.lorawan-architecture} dargestellt.
Die Komponenten sind von links nach rechts mit steigender Abstraktion von der eigentlichen LoRa-Kommunikation angeordnet.
Eine Nachricht vom Endgerät zur Applikation hin wird als \gls{Uplink}, eine Nachricht von der Applikation zum Endgerät hin als \gls{Downlink} bezeichnet.
\cite{lorawanarchitecture}
\image{lorawan-architecture}{img.lorawan-architecture}{\textwidth}{Systemarchitektur bei LoRaWAN}{Systemarchitektur bei LoRaWAN}{bearbeitet aus Zerynth srl, https://www.zerynth.com/wp-content/uploads/2017/05/lorawan-architecture.jpg}

Im folgenden werden die einzelnen Komponenten von LoRaWAN näher erläutert und jeweils deren Verhalten bei einem Up- und Downlink beschrieben.


\subsection{Endgerät}
Endgeräte, auch als \glspl{Node} bezeichnet, werden über das LoRaWAN vernetzt und können Daten zur Applikation senden bzw. von dieser empfangen.
Nodes kommunizieren ausschließlich über LoRa mit dem LoRaWAN, müssen also über die dafür notwendige Hardware (LoRa-Chip, Antenne) verfügen.
Häufig sind Nodes auf besonders energiesparsamen Betrieb ausgelegt und batteriebetrieben.
Um die in \autoref{sec.lora.basics} beschriebenen gesetzlichen Vorgaben einzuhalten, sind die Sendeintervalle vergleichsweise hoch (Minuten bis Stunden oder Tage).

\subsubsection{Uplink}
Ein Node kann Daten per LoRa senden.
Dazu wird auf die Nutzdaten (\gls{Payload}) zuerst die Applikationsverschlüsselung angewendet, darauf wiederum die Netzwerkverschlüsselung.
Per LoRa gesendete Pakete können von allen Nodes und Gateways im Sendebereich empfangen werden.

\subsubsection{Downlink}
Nodes können in verschiedene Geräteklassen eingeteilt werden (siehe \autoref{sec.lorawan.classes}) und sind davon abhängig zu unterschiedlichen Zeiten empfangsbereit.
Wird während eines Empfangsfensters eine an den Node adressierte Nachricht vom Gateway empfangen, so erfolgt die Entfernung der verschiedenen Verschlüsselungsschichten und anschließend die Verarbeitung der Nutzdaten.


\subsection{Gateway}
Ein \gls{Gateway} bildet die Schnittstelle zwischen der Funkkommunikation per LoRa und den restlichen Komponenten des Netzwerks.
LoRa zur Datenübertragung wird somit lediglich zwischen Endgeräten und Gateways genutzt.

In einem LoRaWAN können beliebig viele Gateways genutzt werden.
Durch hinzufügen von Gateways an bisher wenig nicht oder schlecht angebundenen Orten, kann auf einfache Weise die Netzabdeckung verbessert werden.


\subsubsection{Uplink}
Ein Gateway empfängt die von den Endgeräten über LoRa gesendeten Nachrichten.
Außerdem werden Metadaten wie Signalstärke und \gls{SNR} der empfangenen Signale bestimmt.
Die Nutzdaten der LoRa-Nachricht und die Metadaten werden vom Gateway an den Netzwerkserver weitergeleitet, üblicherweise per Internet.

\subsubsection{Downlink}
Ebenso kann das Gateway Daten vom Netzwerkserver empfangen.
Diese werden per LoRa gesendet und können von Endgeräten in Reichweite des Gateways empfangen werden.


\subsection{Netzwerkserver}
Der \gls{Netserver} bildet im LoRaWAN das zentrale Element für das Routing der von Endgeräten empfangenen und an diese zu sendenden Nachrichten. Es gibt genau einen Netzwerkserver im LoRaWAN.

\subsubsection{Uplink}
Der Netzwerkserver verarbeitet die von den Gateways empfangenen Nachrichten.
Wie in \autoref{img.lorawan-architecture} unten links dargestellt, kann die gesendete Nachricht eines Endgeräts von mehreren Gateways gleichzeitig empfangen werden.
Dabei leitet jedes Gateway die Nachricht an den Netzwerkserver weiter.
Durch Vergleich des (teilw. verschlüsselten) Dateninhalts der  Nachricht können mehrfach empfangene Nachrichten im Netzwerkserver zu einer einzigen Nachricht zusammengefasst werden.
Dies wird auch als Deduplizierung bezeichnet.

Außerdem wird für jedes Endgerät gespeichert, über welches Gateway die Signalqualität der empfangenen Daten am besten ist.

Sofern das Endgerät, von dem die Nachricht empfangen wurde, im Netzwerk aktiviert ist (siehe \autoref{sec.lorawan.activation}), kann die Netzwerkverschlüsselung der Nachricht entfernt werden.
Die entschlüsselten Daten werden an den Applikationsserver weitergeleitet, der für dieses Endgerät festgelegt wurde.

\subsubsection{Downlink}
Ebenso kann der Netzwerkserver Downlinkpakete vom Applikationsserver empfangen.
Diese werden zunächst in einer Warteschlange gespeichert (Downlink-Queue).
Abhängig von der Geräteklasse des adressierten Endgeräts erfolgt die weitere Verarbeitung direkt oder durch bestimmte Auslöser (siehe \autoref{sec.lorawan.classes}).
In der Weiterverarbeitung werden die Paketdaten zunächst der Netzwerkverschlüsselung unterzogen.
Außerdem wird für das adressierte Endgerät das Gateway bestimmt, welches aktuell wahrscheinlich die besten Übertragungsbedingungen zum Endgerät hat.
Dies wird anhand der gespeicherten Daten über die Signalqualität vorheriger Uplinks durchgeführt.
Anschließend werden die verschlüsselten Daten an das ermittelte Gateway weitergeleitet.
Ein Paket im Downlink wird also immer nur über genau ein Gateway gesendet.


\subsection{Applikationsserver}
Ein \gls{Appserver} dient der Ent- bzw. Verschlüsselung der Applikationsdaten und ist die Schnittstelle des LoRaWAN zu den eigentlichen Anwendungen.
Im LoRaWAN kann es beliebig viele Applikationsserver geben.

\subsubsection{Uplink}
Ein Applikationsserver empfängt vom Netzwerkserver gesendete Daten.
Dabei wird zunächst die Applikationsverschlüsselung entfernt.
Anschließend sind die eigentlichen vom Endgerät gesendeten Daten (\gls{Payload}) im Klartext verfügbar und können beliebig verarbeitet werden.
Die Metadaten der LoRa-Kommunikation, welche durch die Gateways erfasst wurden (z.\,B. \gls{SNR}), sind ebenso Bestandteil der im Applikationsserver verfügbaren Daten.
Die weitere Verarbeitung der Uplinks ist allerdings nicht mehr Bestandteil der Architektur von LoRaWAN.

\subsubsection{Downlink}
Im Applikationsserver können außerdem Downlinks gestartet werden, also Nachrichten zum Endgerät gesendet werden.
Dabei werden die Daten der Applikationsverschlüsselung unterzogen und das Paket an den Netzwerkserver gesendet.


\subsection{Join-Server}
Damit eine Kommunikation mit Endgeräten im LoRaWAN möglich ist, müssen diese aktiviert werden (siehe \autoref{sec.lorawan.activation}).
Ein Teil des Aktivierungsprozesses erfolgt dabei über den \gls{Joinserver}, insbesondere die Verteilung der Sitzungsschlüssel für die Netzwerk- und Applikationsverschlüsselung.
Im Join-Server wird außerdem der Root-Key für die Verschlüsselung verwaltet, aus dem die genannten Sitzungsschlüssel abgeleitet werden.
Der Join-Server bildet damit ein wichtiges Element zur Sicherstellung der verschlüsselten Kommunikation.
Daher ist dieser insbesondere aus dem Netzwerkserver ausgegliedert, um ein Mitlesen der Applikationsdaten durch den Betreiber des Netzwerkserver zu verhindern.

Im LoRaWAN kann es beliebig viele Join-Server geben, die Zuordnung eines Endgeräts zu einem Join-Server muss allerdings eindeutig sein.



\section{Verschlüsselung}
Nachrichten im LoRaWAN verfügen über mehrere Schichten der Verschlüsselung, um nur denjenigen Komponenten Zugriff auf Daten zu gewähren, wie zum Betrieb des Netzwerks notwendig sind.
Im folgenden sind die Schritte für einen Uplink beschrieben, im Falle eines Downlinks ist die Reihenfolge entsprechend umgekehrt und Ver- und Entschlüsselung sind vertauscht.
Die Anwendungsdaten (\gls{Payload}) werden mit dem 128\,Bit-AES-Applikationssitzungsschlüssel im Endgerät verschlüsselt.
Diese verschlüsselten Daten werden (zusammen mit weiteren für den Netzwerkserver relevanten Daten) im Endgerät mit dem 128\,Bit-AES-Netzwerksitzungsschlüssel verschlüsselt und (ergänzt um zur Adressierung notwendige Daten) per LoRa gesendet.
Geräte, die diese ausgesendeten Daten mithören, haben aufgrund der Verschlüsselung keinen Zugriff auf die Daten, lediglich die Geräteadresse wird unverschlüsselt übertragen.
Auch Gateways können den Inhalt der durch sie weitergeleiteten Pakete nicht mitlesen.
Im Netzwerkserver erfolgt die Entschlüsselung mit dem Netzwerksitzungsschlüssel und der Netzwerkserver erhält Zugriff auf die benötigten Daten.
Die Anwendungsdaten sind allerdings weiterhin verschlüsselt und somit nicht durch den Netzwerkserver lesbar.
Im Applikationsserver erfolgt schlussendlich die Entschlüsselung mit dem Applikationssitzungsschlüssel.
Damit erhält neben dem Endgerät nur der Applikationsserver Zugriff auf die Anwendungsdaten.



\section{Aktivierung}\label{sec.lorawan.activation}
Damit ein Endgerät im LoRaWAN erfolgreich kommunizieren kann, muss es aktiviert sein.
Die Aktivierung dient insbesondere dazu, die für die Ver- und Entschlüsselung notwendigen Sitzungsschlüssel an die zuständigen Komponenten zu verteilen und dem Endgerät eine in diesem LoRaWAN eindeutige Adresse zuzuweisen.
Im folgenden werden die beiden verfügbaren Aktivierungsmethoden beschrieben.
In \autoref{tab.lorawan-activation} sind diese zum Vergleich dargestellt.
\cite{loraactivation}
\begin{singlespacing}
	\begin{table}[htbp]
		\begin{tabular}{l|l|l}
			 & \textbf{Activation by Personalization (ABP)} & \textbf{Over-the-Air Activation (OTAA)} \\ \hline
			\textbf{Sitzung} & eine statische Sitzung & dynamisch, beliebig oft neu \\ \hline
			\textbf{Schlüssel} & \begin{tabular}[c]{@{}l@{}}Sitzungsschlüssel fest\\ einprogrammiert\end{tabular} & \begin{tabular}[c]{@{}l@{}}Aushandlung für Sitzung,\\ Ableitung aus Root-Key\end{tabular} \\ \hline
			\textbf{Beitritt} & Kommunikation direkt möglich & Beitrittsprozess nötig \\ \hline
			\textbf{Sicherheit} & geringer & hoch
		\end{tabular}
		\caption{Vergleich der Methoden für die Aktivierung von Endgeräten}
		\label{tab.lorawan-activation}
	\end{table}
\end{singlespacing}


\subsection{Activation by Personalization (ABP)}
Bei \gls{ABP} werden die Sitzungsschlüssel vor Inbetriebnahme des Geräts berechnet und eine freie Geräteadresse bestimmt.
Diese statischen Sitzungsdaten werden zum einen fest in das Endgerät einprogrammiert und zum anderen an die zuständigen Server verteilt (Netzwerksitzungsschlüssel und Geräteadresse an Netzwerkserver, Applikationssitzungsschlüssel an Applikationsserver).
Damit wurde eine Sitzung etabliert und eine Kommunikation mit diesem Gerät ist sofort möglich.

Nachteil von \gls{ABP} ist die eingeschränkte Sicherheit, da bei Kompromittierung der Schlüssel keine neue Sitzung mit anderen Schlüsseln gestartet werden kann und die Verschlüsselung dann gebrochen ist.
\gls{ABP} wird vor allem beim Debugging während der Entwicklung von Endgeräten eingesetzt, da nach einem Neustart direkt ohne einen Aktivierungsprozess kommuniziert werden kann.


\subsection{Over-the-Air Activation (OTAA)}
Bei \gls{OTAA} erfolgen die Berechnung der Sitzungsschlüssel und die Zuweisung der Geräteadresse dynamisch.
Dabei wird vor Inbetriebnahme des Geräts eine (möglichst) weltweit eindeutige ID für das Gerät (DevEUI) und ein Root-Key festgelegt.
Diese Daten werden im Gerät fest einprogrammiert und im Join-Server hinterlegt.
Im Gerät wird außerdem die ID des zuständigen Join-Servers (JoinEUI) hinterlegt.

Zur Aktivierung wird der in \autoref{img.lorawan-OTAA} dargestellte Beitrittsprozess durchlaufen.
Das Endgerät sendet seine Geräte-ID DevEUI und die ID des Join-Servers JoinEUI unverschlüsselt an das Netzwerk.
Anhand der JoinEUI erfolgt im Netzwerkserver die Weiterleitung an den zuständigen Join-Server.
Dieser wählt eine zufällige Nonce und berechnet daraus anhand des für dieses Endgerät hinterlegten Root-Keys neue Sitzungsschlüssel.
Diese Sitzungsschlüssel werden vom Join-Server an die jeweils dafür zuständigen Server verteilt.
Der Netzwerkserver bestimmt eine für diese Sitzung gültige Adresse für das Endgerät und leitet diese zusammen mit der Nonce des Join-Servers an das Endgerät weiter.
Die Sitzungsschlüssel werden demzufolge nicht per LoRa gesendet und können daher nicht abgehört werden.
Das Endgerät berechnet die Sitzungsschlüssel anhand der Nonce und des Root-Keys über denselben Algorithmus wie der Join-Server.
In der Abbildung sind die Berechnungsschritte in Schritt 4 und 9 demzufolge identisch.
\image{lorawan-OTAA}{img.lorawan-OTAA}{\textwidth}{Beitrittsprozess Over-the-Air Activation}{Beitrittsprozess Over-the-Air Activation}{eigene Darstellung}

Die Sicherheit von \gls{OTAA} ist höher als die von \gls{ABP}, da jederzeit eine neue Sitzung mit neuen Sitzungsschlüsseln initiiert werden kann.
Nachteilig bei \gls{OTAA} ist die Komplexität des Beitrittsprozesses.



\section{Geräteklassen}\label{sec.lorawan.classes}
Endgeräten im LoRaWAN wird eine Geräteklasse zugewiesen.
Anhand dieser Einstellung werden die zeitlichen Intervalle festgelegt, zu denen das Gerät empfangsbereit ist.
In diesen sogenannten Empfangsfenstern kann das Endgerät einen Downlink vom Gateway empfangen, zu allen anderen Zeitpunkten ist ein Empfang durch das Gerät nicht möglich.
Je nach benötigter Latenz einer Nachricht von der Anwendung zum Endgerät wird dem Endgerät die passende Klasse zugewiesen. 
\cite{loraclasses}

\image{lorawan-classes}{img.lorawan-classes}{\textwidth}{Empfangsfenster der Geräteklassen bei LoRaWAN}{Geräteklassen bei LoRaWAN}{https://witekio.com/wp-content/uploads/2018/01/Lora-wan-class.png}
In \autoref{img.lorawan-classes} sind die Empfangsfenster im zeitlichen Verlauf nach einem Uplink des Endgeräts dargestellt.
Allen Klassen gemein ist das Empfangsfenster "`RX1"', welches (abhängig von den Einstellungen des LoRaWAN) üblicherweise 1 Sekunde nach dem Senden einer Nachricht für kurze Zeit geöffnet wird.
In Klasse A und B wird nach einer weiteren kurzen (vom Netzwerk abhängigen) Verzögerung noch das Empfangsfenster "`RX2"' geöffnet.
Danach ist ein Gerät der Klasse A erst wieder nach dem nächsten Uplink erreichbar, die Latenz ist entsprechend am größten und entspricht maximal der Dauer zwischen zwei Uplinks.

Geräte der Klasse B öffnen periodisch ein Empfangsfenster, welches durch vom Gateway gesendete Beacons gesteuert wird.
Die maximale Latenz beträgt hierbei die eingestellte Periodendauer.

Geräte der Klasse C sind (außer wenn sie selber senden) immer empfangsbereit, es gibt entsprechend keine Latenz.

Währends eines offenen Empfangsfensters wird Energie für das Betreiben des LoRa-Chips benötigt.
Der Energiebedarf der Endgeräte ist also abhängig von der eingestellten Klasse.
Bei Klasse C ist der Energiebedarf am höchsten, da der LoRa-Chip dauerhaft aktiviert sein muss.
Der Energiebedarf ist dagegen am geringsten, wenn das Gerät in Klasse A ist, da nur während kurzer Zeitfenster Energie für den Empfang benötigt wird.



\section{Adaptive Data Rate (ADR)}
In \autoref{sec.lora.sf} wurde die Konkurrenz zwischen Airtime, Energiebedarf und Reichweite bei Nutzung verschiedener Spreading Factors beschrieben.
LoRaWAN löst dieses Problem mittels des Verfahrens \gls{ADR}, welches eine Möglichkeit zur automatischen Optimierung dieser Werte bietet.
Die Nutzung des Verfahrens kann durch das Endgerät gesteuert werden und sollte bei stabilen Umgebungsbedingungen (der Funkübertragung) aktiviert werden.
\cite{loraadr}

\image{lorawan-ADR}{img.lorawan-ADR}{\textwidth}{Ablauf ADR}{Ablauf ADR}{eigene Darstellung}
In \autoref{img.lorawan-ADR} ist der Ablauf von ADR dargestellt.
Nach jeder vom Endgerät empfangenen Nachricht wird das vom Gateway ermittelte \acrlong{SNR} durch den Netzwerkserver gespeichert.
Für die Initialisierung werden 20 Nachrichten benötigt.
In der Betriebsphase wird nach jeder empfangenen Nachricht der Durchschnitt des SNR der letzten Nachrichten berechnet.
Anschließend wird die Differenz zum mindestens nötigen SNR für erfolgreichen Empfang gebildet.
Bei einer hohen Differenz gibt es einen SNR-Überschuss und das Endgerät sollte die Sendeleistung reduzieren oder den Spreading Factor verringern (führt zu Erhöhung Datenrate und Verringerung Airtime).
Bei einem SNR-Defizit sollte das Endgerät entsprechend die Sendeleistung oder den Spreading Factor erhöhen (führt zu Verringerung Datenrate und Erhöhung Airtime).
Der Netzwerkserver sendet die passende Kontrollnachricht an das Gerät mit der Information, entsprechend den Spreading Factor zu verändern.

Für die Berechnung wird der Durchschnitt mehrerer Messungen verwendet, damit der Regelkreis nicht durch kurzzeitig auftretende Störungen instabil wird.

\chapter{Anwendungsbeispiele}
Im folgenden Kapitel werden kurze Beispiele zu Anwendungen von LoRa und LoRaWAN vorgestellt.



\section{LoRa}
Für den LoRa-fähigen Mikrocontroller Cubecell\footnote{\url{https://heltec.org/project/htcc-ab02s/}} wurde eine Software entwickelt, mit der die Kommunikation über LoRa getestet werden kann.
Ein Gerät wird dabei als Client genutzt und sendet die per serieller Verbindung übergebenen Daten.
Anschließend versucht das Gerät eine potentielle Antwort zu empfangen.
Ein weiteres Gerät wird als Server genutzt, welcher ständig auf ankommende Pakete wartet.
Wird ein Paket empfangen, so werden die in den Daten enthaltenen Buchstaben zu Großbuchstaben transformiert und per LoRa gesendet.
Durch Client und Server werden die gesendeten und empfangenen Daten sowie die Signalstärke und das \gls{SNR} des empfangenen Pakets jeweils auf dem Bildschirm des Mikrocontrollers dargestellt.

Das Protokoll des Clients ist in \autoref{img.lora-demo-console} dargestellt.
TX bezeichnet dabei die jeweils gesendeten Daten und RX die empfangenen.
In der Demonstration wurde zunächst über wenige Zentimeter Abstand und mit angebrachter Antenne kommuniziert (Nachricht "`Hallo Welt"' und "`Moin"').
Durch Entfernen der Antennen konnte gezeigt werden, dass die Signalstärke  sinkt (Nachricht "`1. Antenne weg"' und "`..."' mit nur einer Antenne, Nachricht "`Antennen weg"' komplett ohne Antennen).
Außerdem wurde die Signalübertragung über mehrere Etagen in den Keller getestet.
Dabei wurde festgestellt, dass ohne Antenne keine Kommunikation möglich ist (erste Nachricht "`Keller?"').
Mit angebrachter Antenne kann problemlos durch mehrere Wände kommuniziert werden, die Signalstärke und das \gls{SNR} sinken allerdings im Vergleich zur Übertragung in über wenige Zentimeter (zweite Nachricht "`Keller?"').
\image{lora-demo-console}{img.lora-demo-console}{0.8\textwidth}{Protokoll des Clients bei Tests der LoRa-Signalübertragung}{Protokoll des Clients bei Tests der LoRa-Signalübertragung}{eigene Darstellung}

Der Quellcode der Software ist auf Github verfügbar.\footnote{\url{https://github.com/Salaml/sensor-networks/tree/main/code}}
Eine Live-Demonstration ist auf Youtube abrufbar.\footnote{\url{https://www.youtube.com/watch?v=YyFAu_R3ZoA}}



\section{LoRaWAN}
In der HTW Dresden wurde ein LoRaWAN-Gateway installiert und in das freie LoRaWAN \gls{TTN} eingebunden.
Im folgenden wird kurz auf einige durch das Gateway bereitgestellte bzw. erfasste Statistiken eingegangen.

In \autoref{img.gateway-duty-cycle} ist der Duty Cycle des Gateways nach Kanal aufgeschlüsselt dargestellt.
Es ist erkennbar, dass einige Kanäle häufiger verwendet werden als andere.
Außerdem wird deutlich, dass LoRa-Übertragungen nicht dauerhaft stattfinden sondern aufgrund des gesetzlichen begrenzten Duty Cycles nur zu sporadischen Zeitpunkten.
\image{gateway-duty-cycle}{img.gateway-duty-cycle}{0.6\textwidth}{Duty Cycle des LoRaWAN-Gateways der HTW Dresden nach Kanal}{Duty Cycle des LoRaWAN-Gateways der HTW Dresden nach Kanal}{eigene Darstellung}

In \autoref{img.gateway-range} ist die Reichweite des Gateways erkennbar.
\image{gateway-range}{img.gateway-range}{\textwidth}{kartierte Signalstärke des LoRaWAN-Gateways der HTW Dresden}{kartierte Signalstärke des LoRaWAN-Gateways der HTW Dresden}{TTN-Mapper, https://ttnmapper.org/heatmap/private/?gateway=htw-dresden-ttn-gw1\&network=NS\_TTS\_V3://ttn@000013}
Die Punkte sind nach der Signalstärke gefärbt, mit der eine Nachricht vom angegebenen Punkt das Gateway erreicht hat.
Die Daten werden erfasst, indem ein Node periodisch (beliebige) Daten ins LoRaWAN sendet während er mit dem Kartierenden in der Umwelt bewegt wird.
Die empfangenen Daten werden durch das Mobiltelefon des Kartierenden vom Applikationsserver über eine MQTT-Schnittstelle abgerufen.
Das Mobiltelefon kombiniert die aus den Metadaten verfügbare Signalstärke zusammen mit der aktuellen Position und sendet diese dann an den Server des TTN-Mapper.
Dort werden die Daten aufbereitet und entsprechend auf der Karte dargestellt.
Dieser Workflow demonstriert perfekt den Datenfluss durch ein LoRaWAN.


\input{zusammenfassung}

\clearpage
\printbibliography[title=Literaturverzeichnis]

\end{document}