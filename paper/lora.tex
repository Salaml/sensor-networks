\chapter{LoRa}\label{chp.lora}



\section{Grundlagen}\label{sec.lora.basics}
LoRa ist ein proprietäres Übertragungsverfahren per Funk.
Es wurde vom Unternehmen Semtech entwickelt, um Kommunikation mit niedrigem Energiebedarf über große Distanzen zu ermöglichen.
Im Vergleich zu anderen Funktechnologien wie Bluetooth oder Mobilfunk, ist die "`Bandbreite"' der Datenübertragung (Datenrate) bei LoRa nur gering.
Dagegen ist die erzielbare Reichweite größer, wie \autoref{img.bandwidth-range} zeigt.
\cite{semtech}

\image{bandwidth-vs-range}{img.bandwidth-range}{\textwidth}{Bandbreite (Datenrate) und Reichweite von LoRa im Vergleich zu anderen Funktechnologien}{Bandbreite und Reichweite von LoRa im Vergleich zu anderen Funktechnologien}{The Things Industries, https://www.thethingsnetwork.org/docs/lorawan/what-is-lorawan/bandwidth-vs-range.png}

Für LoRa werden ortsabhängig Frequenzbereiche aus lizenzfreien Bändern genutzt (\gls{ISM}, \gls{SRD}).
In Europa kommt das SRD-Band um 868\,MHz zum Einsatz.\cite{loraparameter}
Obwohl es sich hierbei um ein freies Frequenzband handelt, sind dennoch einige Einschränkungen zu beachten:\cite{lorabasics}
\begin{itemize}\singlespacing\setlength\itemsep{-0.2em}
\item Sendeleistung \gls{Uplink} begrenzt auf 25\,mW (14\,dBm)
\item Sendeleistung \gls{Downlink} begrenzt auf 500\,mW (27\,dBm)
\item Duty Cycle Senden pro Gerät begrenzt auf kanalabhängig 0,1\% bzw. 1\%
\end{itemize}
Diese Grenzen sind gesetzlich festgelegt, um eine Blockade des Bandes unter anderem durch zu häufiges Senden zu verhindern.
Damit ist jedem eine faire Verwendung dieses Bandes unter Beachtung der Grenzen möglich.

Um insbesondere die Begrenzung der Sendezeit inzuhalten zu können, sollte eine Minimierung der Sendezeit (\gls{Airtime}) erfolgen.



\section{Modulation Chirp Spread Spectrum }\label{sec.lora.modulation}
Per LoRa gesendete Signale werden mittels des Verfahrens \gls{Modulation} moduliert.
Grundlage dieser Modulation sind sogenannte \glspl{Chirp}.
Eine Anpassung der Datenrate ist über den sogenannten \gls{SF} möglich.


\subsection{Chirps}\label{sec.lora.modulation.chirp}
Ein Chirp ist ein Impuls mit konstanter Amplitude und sich ändernder Frequenz.
Bei steigender Frequenz wird der Chirp als Upchirp bezeichnet, bei fallender als Downchirp.
In \autoref{img.chirp-linear} ist ein Upchirp mit linearem Anstieg der Frequenz veranschaulicht.
\image{chirp-linear}{img.chirp-linear}{0.7\textwidth}{Upchirp mit linearem Frequenzanstieg}{Upchirp mit linearem Frequenzanstieg}{Georg-Johann, CC BY-SA 3.0, via Wikimedia Commons, https://commons.wikimedia.org/wiki/File:Linear-chirp.svg}

Die Änderung der Frequenz der Chirps erfolgt nicht kontinuierlich, sondern zu diskreten Zeitpunkten.
Ein einzelner Zustand heißt dabei Chip, die Anzahl an Frequenzänderungen pro Zeiteinheit wird als Chiprate $R_c$ bezeichnet.
Bei LoRa entspricht dies der von den Chirps genutzten Bandbreite $B$ (üblicherweise 125\,kHz).
\cite[S. 10]{loramodulation}
\begin{equation}
	R_c = B = 125\,\mathrm{kHz}
	\label{eq.chip-rate}
\end{equation}


\subsection{Symbole}
Zur Kodierung von Daten wird ein Chirp zeitlich verschoben und bildet damit ein sogenanntes Symbol.
Die Startfrequenz des Symbols legt dabei die kodierten Daten fest.
Nachdem die Frequenz des Chirps innerhalb des Symbols die Grenze der Bandbreite erreicht hat, wird der Chirp mit der Frequenz der anderen Bereichsgrenze fortgesetzt (Upchirp: Sprung von $f_{max}$ zu $f_{min}$, Downchirp: Sprung von $f_{min}$ zu $f_{max}$).
\cite{lorachirps}

\image{lora-symbols}{img.lora-symbols}{\textwidth}{Spektrum verschiedener LoRa-Symbole}{Spektrum verschiedener LoRa-Symbole}{bearbeitet aus Sakshama Ghoslya, https://www.sghoslya.com/p/lora\_9.html}
In \autoref{img.lora-symbols} ist der Frequenzgang verschiedener Symbole anhand des Spektrums veranschaulicht.
Innerhalb der Symbole ist jeweils der Sprung von der maximalen Frequenz $f_{max}$ zur minimalen Frequenz $f_{min}$ und damit der Lauf über den kompletten Frequenzbereich erkennbar.
Innerhalb eines Symbols wird dementsprechend immer die komplette Bandbreite genutzt, lediglich die Startfrequenz unterscheidet sich nach den kodierten Daten.
Aufgrund dieser Verteilung des Signals über das Spektrum wird das Modulationsverfahren Chirp Spread Spectrum genannt.

Die Anzahl der Symbole pro Zeiteinheit wird als Symbolrate $R_s$ bezeichnet.


\subsection{Spreading Factor}
Die Dauer und der Informationsgehalt der Symbole kann über denn sogenannten Spreading Factor $SF$ eingestellt werden.
Dieser gibt die Anzahl Bits an, die in einem Symbol kodiert werden können.
Für die Symbolrate gilt die in \autoref{eq.symbol-rate} dargestellte Beziehung.
\cite[S. 10]{loramodulation}
\begin{equation}
	R_s = \frac{B}{2^{SF}} = \frac{R_c}{2^{SF}}
	\label{eq.symbol-rate}
\end{equation}
Bei Erhöhung des Spreading Factors um 1 halbiert sich damit die Symbolrate, die Dauer der Symbole verdoppelt sich.
In \autoref{img.lora-spreading-factors} ist die Verdopplung der Zeitdauer bei Erhöhung des Spreading Factors deutlich erkennbar.
\image{lora-spreading-factors}{img.lora-spreading-factors}{\textwidth}{Spektrum verschiedener Spreading Factors}{Spektrum verschiedener Spreading Factors}{bearbeitet aus Sakshama Ghoslya, https://www.sghoslya.com/p/lora-is-chirp-spread-spectrum.html}

% TODO Beziehung zu Chirprate
% TODO coding rate -> 4 Bits + 1 bis 4 Bits
% R_b = SF * BW / 2^SF
% R_b = SF * (4 / 4+CR) / (2^SF / BW)

LoRa-Empfänger weisen eine bessere Empfindlichkeit gegenüber Signalen mit höherem Spreading Factor auf, wie in \autoref{tab.spreading-factors} dargestellt.
\begin{table}[htbp]
\centering
\begin{tabular}{@{}lcr@{}}
\textbf{Spreading Factor} & \textbf{Empfindlichkeit} & \textbf{Airtime} \\
SF7  & -123,0\,dBm &  41\,ms \\
SF8  & -126,0\,dBm &  72\,ms \\
SF9  & -129,0\,dBm & 144\,ms \\
SF10 & -132,0\,dBm & 288\,ms \\
SF11 & -134,5\,dBm & 577\,ms \\
SF12 & -137,0\,dBm & 991\,ms \\
\end{tabular}
\caption[Empfindlichkeit und Airtime verschiedener Spreading Factors]{Empfindlichkeit und Airtime verschiedener Spreading Factors\\Quelle: Semtech Corporation, https://lora-developers.semtech.com/documentation/tech-papers-and-guides/understanding-adr}
\label{tab.spreading-factors}
\end{table}
Eine Erhöhung des Spreading Factors führt damit zu folgenden Effekten:
\begin{itemize}\singlespacing\setlength\itemsep{-0.2em}
\item Airtime Sender erhöht
\item Energiebedarf Sender erhöht
\item Empfindlichkeit Empfänger verbessert
\item Übertragungsreichweite erhöht
\end{itemize}
Für diese Werte gelten jedoch unterschiedliche Ziele: Airtime und Energiebedarf sollten minimiert werden, Empfindlichkeit und Reichweite maximiert werden.\cite{loralimits}
Um einen bestmöglichen Kompromiss zwischen diesen Vorgaben zu schaffen, implementiert LoRaWAN die in \autoref{sec.lorawan.adr} beschriebene automatische Anpassung des Spreading Factors.


\subsection{Orthogonalität}
TODO


\section{LoRa-Nachricht}
Eine per LoRa gesendete Nachricht besteht aus den folgenden Elementen:\cite{loralimits}
\begin{itemize}\singlespacing\setlength\itemsep{-0.2em}
\item Präambel
\item Header + \gls{CRC} für Header
\item Nutzdaten
\item \gls{CRC}
\end{itemize}

Die Präambel dient zur Synchronisation des Empfängers auf die gesendeten Daten.
Anhand der Präambel wird außerdem der Spreading Factor erkannt, mit welcher die Daten gesendet wurden.

Der Header enthält unter anderem Informationen zur Länge der Nutzdaten.

Am Ende der Nachricht folgt optional ein CRC-Wert zu Fehlerdetektion bzw. -korrektur.